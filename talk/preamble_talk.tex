%!TEX root = talk.tex


% Fonts and encoding
\usepackage[utf8]{inputenc} % allow utf-8 input
\usepackage[T1]{fontenc}    % use 8-bit T1 fonts
\usepackage{microtype}      % microtypography
\usepackage{inconsolata}
\usepackage[sfdefault,light,condensed]{roboto}
\usefonttheme[onlymath]{serif} % serif fonts for mathematics

% Enumerations and Items
\usepackage{fourier-orns} % starred bullets
% \usepackage{enumitem}


% Mathematics and Notation
\usepackage{amsmath}
\usepackage{amssymb}
\usepackage{amsfonts}       % blackboard math symbols
\usepackage{nicefrac}       % compact symbols for 1/2, etc.
\usepackage{mathtools}
\usepackage{bm}
\usepackage{physics}
% (Adjusted) mathematics commands from https://github.com/goodfeli/dlbook_notation.
%%%%% NEW MATH DEFINITIONS %%%%%

\usepackage{amsmath,amsfonts,bm}

% Mark sections of captions for referring to divisions of figures
\newcommand{\figleft}{{\em (Left)}}
\newcommand{\figcenter}{{\em (Center)}}
\newcommand{\figright}{{\em (Right)}}
\newcommand{\figtop}{{\em (Top)}}
\newcommand{\figbottom}{{\em (Bottom)}}
\newcommand{\captiona}{{\em (a)}}
\newcommand{\captionb}{{\em (b)}}
\newcommand{\captionc}{{\em (c)}}
\newcommand{\captiond}{{\em (d)}}

% Highlight a newly defined term
\newcommand{\newterm}[1]{{\bf #1}}


% Figure reference, lower-case.
\def\figref#1{figure~\ref{#1}}
% Figure reference, capital. For start of sentence
\def\Figref#1{Figure~\ref{#1}}
\def\twofigref#1#2{figures \ref{#1} and \ref{#2}}
\def\quadfigref#1#2#3#4{figures \ref{#1}, \ref{#2}, \ref{#3} and \ref{#4}}
% Section reference, lower-case.
\def\secref#1{section~\ref{#1}}
% Section reference, capital.
\def\Secref#1{Section~\ref{#1}}
% Reference to two sections.
\def\twosecrefs#1#2{sections \ref{#1} and \ref{#2}}
% Reference to three sections.
\def\secrefs#1#2#3{sections \ref{#1}, \ref{#2} and \ref{#3}}
% Reference to an equation, lower-case.
%\def\eqref#1{equation~\ref{#1}}
% Reference to an equation, upper case
%\def\Eqref#1{Equation~\ref{#1}}
% A raw reference to an equation---avoid using if possible
\def\plaineqref#1{\ref{#1}}
% Reference to a chapter, lower-case.
\def\chapref#1{chapter~\ref{#1}}
% Reference to an equation, upper case.
\def\Chapref#1{Chapter~\ref{#1}}
% Reference to a range of chapters
\def\rangechapref#1#2{chapters\ref{#1}--\ref{#2}}
% Reference to an algorithm, lower-case.
\def\algref#1{algorithm~\ref{#1}}
% Reference to an algorithm, upper case.
\def\Algref#1{Algorithm~\ref{#1}}
\def\twoalgref#1#2{algorithms \ref{#1} and \ref{#2}}
\def\Twoalgref#1#2{Algorithms \ref{#1} and \ref{#2}}
% Reference to a part, lower case
\def\partref#1{part~\ref{#1}}
% Reference to a part, upper case
\def\Partref#1{Part~\ref{#1}}
\def\twopartref#1#2{parts \ref{#1} and \ref{#2}}

\def\ceil#1{\lceil #1 \rceil}
\def\floor#1{\lfloor #1 \rfloor}
\def\1{\bm{1}}
\newcommand{\train}{\mathcal{D}}
\newcommand{\valid}{\mathcal{D_{\mathrm{valid}}}}
\newcommand{\test}{\mathcal{D_{\mathrm{test}}}}

\def\eps{{\epsilon}}


% Random variables
\def\reta{{\mathsf{$\eta$}}}
\def\ra{{\mathsf{a}}}
\def\rb{{\mathsf{b}}}
\def\rc{{\mathsf{c}}}
\def\rd{{\mathsf{d}}}
\def\re{{\mathsf{e}}}
\def\rf{{\mathsf{f}}}
\def\rg{{\mathsf{g}}}
\def\rh{{\mathsf{h}}}
\def\ri{{\mathsf{i}}}
\def\rj{{\mathsf{j}}}
\def\rk{{\mathsf{k}}}
\def\rl{{\mathsf{l}}}
% rm is already a command, just don't name any random variables m
\def\rn{{\mathsf{n}}}
\def\ro{{\mathsf{o}}}
\def\rp{{\mathsf{p}}}
\def\rq{{\mathsf{q}}}
\def\rr{{\mathsf{r}}}
\def\rs{{\mathsf{s}}}
\def\rt{{\mathsf{t}}}
\def\ru{{\mathsf{u}}}
\def\rv{{\mathsf{v}}}
\def\rw{{\mathsf{w}}}
\def\rx{{\mathsf{x}}}
\def\ry{{\mathsf{y}}}
\def\rz{{\mathsf{z}}}

% Random vectors
\def\rvepsilon{{\bm{\mathsf{\epsilon}}}}
\def\rvtheta{{\bm{\mathsf{\theta}}}}
\def\rva{{\bm{\mathsf{a}}}}
\def\rvb{{\bm{\mathsf{b}}}}
\def\rvc{{\bm{\mathsf{c}}}}
\def\rvd{{\bm{\mathsf{d}}}}
\def\rve{{\bm{\mathsf{e}}}}
\def\rvf{{\bm{\mathsf{f}}}}
\def\rvg{{\bm{\mathsf{g}}}}
\def\rvh{{\bm{\mathsf{h}}}}
\def\rvu{{\bm{\mathsf{i}}}}
\def\rvj{{\bm{\mathsf{j}}}}
\def\rvk{{\bm{\mathsf{k}}}}
\def\rvl{{\bm{\mathsf{l}}}}
\def\rvm{{\bm{\mathsf{m}}}}
\def\rvn{{\bm{\mathsf{n}}}}
\def\rvo{{\bm{\mathsf{o}}}}
\def\rvp{{\bm{\mathsf{p}}}}
\def\rvq{{\bm{\mathsf{q}}}}
\def\rvr{{\bm{\mathsf{r}}}}
\def\rvs{{\bm{\mathsf{s}}}}
\def\rvt{{\bm{\mathsf{t}}}}
\def\rvu{{\bm{\mathsf{u}}}}
\def\rvv{{\bm{\mathsf{v}}}}
\def\rvw{{\bm{\mathsf{w}}}}
\def\rvx{{\bm{\mathsf{x}}}}
\def\rvy{{\bm{\mathsf{y}}}}
\def\rvz{{\bm{\mathsf{z}}}}

% Elements of random vectors
\def\erva{\bm{\mathsf{a}}}
\def\ervb{\bm{\mathsf{b}}}
\def\ervc{\bm{\mathsf{c}}}
\def\ervd{{\mathsf{d}}}
\def\erve{{\mathsf{e}}}
\def\ervf{{\mathsf{f}}}
\def\ervg{{\mathsf{g}}}
\def\ervh{{\mathsf{h}}}
\def\ervi{{\mathsf{i}}}
\def\ervj{{\mathsf{j}}}
\def\ervk{{\mathsf{k}}}
\def\ervl{{\mathsf{l}}}
\def\ervm{{\mathsf{m}}}
\def\ervn{{\mathsf{n}}}
\def\ervo{{\mathsf{o}}}
\def\ervp{{\mathsf{p}}}
\def\ervq{{\mathsf{q}}}
\def\ervr{{\mathsf{r}}}
\def\ervs{{\mathsf{s}}}
\def\ervt{{\mathsf{t}}}
\def\ervu{{\mathsf{u}}}
\def\ervv{{\mathsf{v}}}
\def\ervw{{\mathsf{w}}}
\def\ervx{{\mathsf{x}}}
\def\ervy{{\mathsf{y}}}
\def\ervz{{\mathsf{z}}}

% Random matrices
\def\rmA{{\bm{\mathsf{A}}}}
\def\rmB{{\bm{\mathsf{B}}}}
\def\rmC{{\bm{\mathsf{C}}}}
\def\rmD{{\bm{\mathsf{D}}}}
\def\rmE{{\bm{\mathsf{E}}}}
\def\rmF{{\bm{\mathsf{F}}}}
\def\rmG{{\bm{\mathsf{G}}}}
\def\rmH{{\bm{\mathsf{H}}}}
\def\rmI{{\bm{\mathsf{I}}}}
\def\rmJ{{\bm{\mathsf{J}}}}
\def\rmK{{\bm{\mathsf{K}}}}
\def\rmL{{\bm{\mathsf{L}}}}
\def\rmM{{\bm{\mathsf{M}}}}
\def\rmN{{\bm{\mathsf{N}}}}
\def\rmO{{\bm{\mathsf{O}}}}
\def\rmP{{\bm{\mathsf{P}}}}
\def\rmQ{{\bm{\mathsf{Q}}}}
\def\rmR{{\bm{\mathsf{R}}}}
\def\rmS{{\bm{\mathsf{S}}}}
\def\rmT{{\bm{\mathsf{T}}}}
\def\rmU{{\bm{\mathsf{U}}}}
\def\rmV{{\bm{\mathsf{V}}}}
\def\rmW{{\bm{\mathsf{W}}}}
\def\rmX{{\bm{\mathsf{X}}}}
\def\rmY{{\bm{\mathsf{Y}}}}
\def\rmZ{{\bm{\mathsf{Z}}}}
\def\rmBeta{{\bm{\mathsf{\beta}}}}
\def\rmPi{{\bm{\mathsf{\Pi}}}}
\def\rmPhi{{\bm{\mathsf{\Phi}}}}
\def\rmPsi{{\bm{\mathsf{\Psi}}}}
\def\rmLambda{{\bm{\mathsf{\Lambda}}}}
\def\rmSigma{{\bm{\mathsf{\Sigma}}}}
\def\rmDelta{{\bm{\mathsf{\Delta}}}}

% Elements of random matrices
\def\ermA{{\mathsf{A}}}
\def\ermB{{\mathsf{B}}}
\def\ermC{{\mathsf{C}}}
\def\ermD{{\mathsf{D}}}
\def\ermE{{\mathsf{E}}}
\def\ermF{{\mathsf{F}}}
\def\ermG{{\mathsf{G}}}
\def\ermH{{\mathsf{H}}}
\def\ermI{{\mathsf{I}}}
\def\ermJ{{\mathsf{J}}}
\def\ermK{{\mathsf{K}}}
\def\ermL{{\mathsf{L}}}
\def\ermM{{\mathsf{M}}}
\def\ermN{{\mathsf{N}}}
\def\ermO{{\mathsf{O}}}
\def\ermP{{\mathsf{P}}}
\def\ermQ{{\mathsf{Q}}}
\def\ermR{{\mathsf{R}}}
\def\ermS{{\mathsf{S}}}
\def\ermT{{\mathsf{T}}}
\def\ermU{{\mathsf{U}}}
\def\ermV{{\mathsf{V}}}
\def\ermW{{\mathsf{W}}}
\def\ermX{{\mathsf{X}}}
\def\ermY{{\mathsf{Y}}}
\def\ermZ{{\mathsf{Z}}}

% Vectors
\def\vzero{{\bm{0}}}
\def\vone{{\bm{1}}}
\def\vmu{{\bm{\mu}}}
\def\vtheta{{\bm{\theta}}}
\def\valpha{{\bm{\alpha}}}
\def\vbeta{{\bm{\beta}}}
\def\vgamma{{\bm{\gamma}}}
\def\vlambda{{\bm{\lambda}}}
\def\vphi{{\bm{\phi}}}
\def\vpsi{{\bm{\psi}}}
\def\vxi{{\bm{\xi}}}
\def\va{{\bm{a}}}
\def\vb{{\bm{b}}}
\def\vc{{\bm{c}}}
\def\vd{{\bm{d}}}
\def\ve{{\bm{e}}}
\def\vf{{\bm{f}}}
\def\vg{{\bm{g}}}
\def\vh{{\bm{h}}}
\def\vi{{\bm{i}}}
\def\vj{{\bm{j}}}
\def\vk{{\bm{k}}}
\def\vl{{\bm{l}}}
\def\vm{{\bm{m}}}
\def\vn{{\bm{n}}}
\def\vo{{\bm{o}}}
\def\vp{{\bm{p}}}
\def\vq{{\bm{q}}}
\def\vr{{\bm{r}}}
\def\vs{{\bm{s}}}
\def\vt{{\bm{t}}}
\def\vu{{\bm{u}}}
\def\vv{{\bm{v}}}
\def\vw{{\bm{w}}}
\def\vx{{\bm{x}}}
\def\vy{{\bm{y}}}
\def\vz{{\bm{z}}}

% Elements of vectors
\def\evalpha{{\alpha}}
\def\evbeta{{\beta}}
\def\evepsilon{{\epsilon}}
\def\evlambda{{\lambda}}
\def\evomega{{\omega}}
\def\evmu{{\mu}}
\def\evpsi{{\psi}}
\def\evsigma{{\sigma}}
\def\evtheta{{\theta}}
\def\evphi{{\phi}}
\def\evxi{{\xi}}
\def\eva{{a}}
\def\evb{{b}}
\def\evc{{c}}
\def\evd{{d}}
\def\eve{{e}}
\def\evf{{f}}
\def\evg{{g}}
\def\evh{{h}}
\def\evi{{i}}
\def\evj{{j}}
\def\evk{{k}}
\def\evl{{l}}
\def\evm{{m}}
\def\evn{{n}}
\def\evo{{o}}
\def\evp{{p}}
\def\evq{{q}}
\def\evr{{r}}
\def\evs{{s}}
\def\evt{{t}}
\def\evu{{u}}
\def\evv{{v}}
\def\evw{{w}}
\def\evx{{x}}
\def\evy{{y}}
\def\evz{{z}}

% Matrix
\def\mA{{\bm{A}}}
\def\mB{{\bm{B}}}
\def\mC{{\bm{C}}}
\def\mD{{\bm{D}}}
\def\mE{{\bm{E}}}
\def\mF{{\bm{F}}}
\def\mG{{\bm{G}}}
\def\mH{{\bm{H}}}
\def\mI{{\bm{I}}}
\def\mJ{{\bm{J}}}
\def\mK{{\bm{K}}}
\def\mL{{\bm{L}}}
\def\mM{{\bm{M}}}
\def\mN{{\bm{N}}}
\def\mO{{\bm{O}}}
\def\mP{{\bm{P}}}
\def\mQ{{\bm{Q}}}
\def\mR{{\bm{R}}}
\def\mS{{\bm{S}}}
\def\mT{{\bm{T}}}
\def\mU{{\bm{U}}}
\def\mV{{\bm{V}}}
\def\mW{{\bm{W}}}
\def\mX{{\bm{X}}}
\def\mY{{\bm{Y}}}
\def\mZ{{\bm{Z}}}
\def\mBeta{{\bm{\beta}}}
\def\mPi{{\bm{\Pi}}}
\def\mPhi{{\bm{\Phi}}}
\def\mPsi{{\bm{\Psi}}}
\def\mLambda{{\bm{\Lambda}}}
\def\mSigma{{\bm{\Sigma}}}
\def\mDelta{{\bm{\Delta}}}

% Tensor
\DeclareMathAlphabet{\mathsfit}{\encodingdefault}{\sfdefault}{m}{sl}
\SetMathAlphabet{\mathsfit}{bold}{\encodingdefault}{\sfdefault}{bx}{n}
\newcommand{\tens}[1]{\bm{\mathsfit{#1}}}
\def\tA{{\tens{A}}}
\def\tB{{\tens{B}}}
\def\tC{{\tens{C}}}
\def\tD{{\tens{D}}}
\def\tE{{\tens{E}}}
\def\tF{{\tens{F}}}
\def\tG{{\tens{G}}}
\def\tH{{\tens{H}}}
\def\tI{{\tens{I}}}
\def\tJ{{\tens{J}}}
\def\tK{{\tens{K}}}
\def\tL{{\tens{L}}}
\def\tM{{\tens{M}}}
\def\tN{{\tens{N}}}
\def\tO{{\tens{O}}}
\def\tP{{\tens{P}}}
\def\tQ{{\tens{Q}}}
\def\tR{{\tens{R}}}
\def\tS{{\tens{S}}}
\def\tT{{\tens{T}}}
\def\tU{{\tens{U}}}
\def\tV{{\tens{V}}}
\def\tW{{\tens{W}}}
\def\tX{{\tens{X}}}
\def\tY{{\tens{Y}}}
\def\tZ{{\tens{Z}}}


% Graph
\def\gA{{\mathcal{A}}}
\def\gB{{\mathcal{B}}}
\def\gC{{\mathcal{C}}}
\def\gD{{\mathcal{D}}}
\def\gE{{\mathcal{E}}}
\def\gF{{\mathcal{F}}}
\def\gG{{\mathcal{G}}}
\def\gH{{\mathcal{H}}}
\def\gI{{\mathcal{I}}}
\def\gJ{{\mathcal{J}}}
\def\gK{{\mathcal{K}}}
\def\gL{{\mathcal{L}}}
\def\gM{{\mathcal{M}}}
\def\gN{{\mathcal{N}}}
\def\gO{{\mathcal{O}}}
\def\gP{{\mathcal{P}}}
\def\gQ{{\mathcal{Q}}}
\def\gR{{\mathcal{R}}}
\def\gS{{\mathcal{S}}}
\def\gT{{\mathcal{T}}}
\def\gU{{\mathcal{U}}}
\def\gV{{\mathcal{V}}}
\def\gW{{\mathcal{W}}}
\def\gX{{\mathcal{X}}}
\def\gY{{\mathcal{Y}}}
\def\gZ{{\mathcal{Z}}}

% Sets
\def\sA{{\mathbb{A}}}
\def\sB{{\mathbb{B}}}
\def\sC{{\mathbb{C}}}
\def\sD{{\mathbb{D}}}
% Don't use a set called E, because this would be the same as our symbol
% for expectation.
\def\sF{{\mathbb{F}}}
\def\sG{{\mathbb{G}}}
\def\sH{{\mathbb{H}}}
\def\sI{{\mathbb{I}}}
\def\sJ{{\mathbb{J}}}
\def\sK{{\mathbb{K}}}
\def\sL{{\mathbb{L}}}
\def\sM{{\mathbb{M}}}
\def\sN{{\mathbb{N}}}
\def\sO{{\mathbb{O}}}
\def\sP{{\mathbb{P}}}
\def\sQ{{\mathbb{Q}}}
\def\sR{{\mathbb{R}}}
\def\sS{{\mathbb{S}}}
\def\sT{{\mathbb{T}}}
\def\sU{{\mathbb{U}}}
\def\sV{{\mathbb{V}}}
\def\sW{{\mathbb{W}}}
\def\sX{{\mathbb{X}}}
\def\sY{{\mathbb{Y}}}
\def\sZ{{\mathbb{Z}}}

% Entries of a matrix
\def\emLambda{{\Lambda}}
\def\emA{{A}}
\def\emB{{B}}
\def\emC{{C}}
\def\emD{{D}}
\def\emE{{E}}
\def\emF{{F}}
\def\emG{{G}}
\def\emH{{H}}
\def\emI{{I}}
\def\emJ{{J}}
\def\emK{{K}}
\def\emL{{L}}
\def\emM{{M}}
\def\emN{{N}}
\def\emO{{O}}
\def\emP{{P}}
\def\emQ{{Q}}
\def\emR{{R}}
\def\emS{{S}}
\def\emT{{T}}
\def\emU{{U}}
\def\emV{{V}}
\def\emW{{W}}
\def\emX{{X}}
\def\emY{{Y}}
\def\emZ{{Z}}
\def\emSigma{{\Sigma}}

% entries of a tensor
% Same font as tensor, without \bm wrapper
\newcommand{\etens}[1]{\mathsfit{#1}}
\def\etLambda{{\etens{\Lambda}}}
\def\etA{{\etens{A}}}
\def\etB{{\etens{B}}}
\def\etC{{\etens{C}}}
\def\etD{{\etens{D}}}
\def\etE{{\etens{E}}}
\def\etF{{\etens{F}}}
\def\etG{{\etens{G}}}
\def\etH{{\etens{H}}}
\def\etI{{\etens{I}}}
\def\etJ{{\etens{J}}}
\def\etK{{\etens{K}}}
\def\etL{{\etens{L}}}
\def\etM{{\etens{M}}}
\def\etN{{\etens{N}}}
\def\etO{{\etens{O}}}
\def\etP{{\etens{P}}}
\def\etQ{{\etens{Q}}}
\def\etR{{\etens{R}}}
\def\etS{{\etens{S}}}
\def\etT{{\etens{T}}}
\def\etU{{\etens{U}}}
\def\etV{{\etens{V}}}
\def\etW{{\etens{W}}}
\def\etX{{\etens{X}}}
\def\etY{{\etens{Y}}}
\def\etZ{{\etens{Z}}}

% The true underlying data generating distribution
\newcommand{\pdata}{p_{\rm{data}}}
% The empirical distribution defined by the training set
\newcommand{\ptrain}{\hat{p}_{\rm{data}}}
\newcommand{\Ptrain}{\hat{P}_{\rm{data}}}
% The model distribution
\newcommand{\pmodel}{p_{\rm{model}}}
\newcommand{\Pmodel}{P_{\rm{model}}}
\newcommand{\ptildemodel}{\tilde{p}_{\rm{model}}}
% Stochastic autoencoder distributions
\newcommand{\pencode}{p_{\rm{encoder}}}
\newcommand{\pdecode}{p_{\rm{decoder}}}
\newcommand{\precons}{p_{\rm{reconstruct}}}

\newcommand{\laplace}{\mathrm{Laplace}} % Laplace distribution

\newcommand{\E}{\mathbb{E}}
\newcommand{\Ls}{\mathcal{L}}
\newcommand{\R}{\mathbb{R}}
\newcommand{\emp}{\tilde{p}}
\newcommand{\lr}{\alpha}
\newcommand{\reg}{\lambda}
\newcommand{\rect}{\mathrm{rectifier}}
\newcommand{\softmax}{\mathrm{softmax}}
\newcommand{\sigmoid}{\sigma}
\newcommand{\softplus}{\zeta}
\newcommand{\KL}{D_{\mathrm{KL}}}
\newcommand{\Var}{\mathrm{Var}}
\newcommand{\standarderror}{\mathrm{SE}}
\newcommand{\Cov}{\mathrm{Cov}}
% Wolfram Mathworld says $L^2$ is for function spaces and $\ell^2$ is for vectors
% But then they seem to use $L^2$ for vectors throughout the site, and so does
% wikipedia.
\newcommand{\normlzero}{L^0}
\newcommand{\normlone}{L^1}
\newcommand{\normltwo}{L^2}
\newcommand{\normlp}{L^p}
\newcommand{\normmax}{L^\infty}

\newcommand{\parents}{Pa} % See usage in notation.tex. Chosen to match Daphne's book.

\DeclareMathOperator*{\argmax}{arg\,max}
\DeclareMathOperator*{\argmin}{arg\,min}

\DeclareMathOperator{\sign}{sign}
%\DeclareMathOperator{\Tr}{Tr}
\let\ab\allowbreak



% Figures
\usepackage{graphicx}
\usepackage{pgfplots}
\pgfplotsset{compat=newest}
\usepgfplotslibrary{groupplots}
\usepgfplotslibrary{dateplot}
\usepackage{caption}
\captionsetup[table]{skip=5pt}
\usepackage{subcaption}
%\usepackage{subfigure}

% Plot dimensions
\newlength{\figureheight}
\newlength{\figurewidth}
\newlength{\figheight}
\newlength{\figwidth}


% Algorithms
\usepackage{algorithm}% http://ctan.org/pkg/algorithm
\usepackage{algpseudocode}% http://ctan.org/pkg/algorithmicx
\algrenewcommand{\algorithmiccomment}[1]{\hfill {\textcolor{darkgray}{\# #1}}}
%\algrenewcommand\alglinenumber[1]{\tiny #1}


% Theorem-type environments
% \usepackage{amsthm}
% \newtheoremstyle{theorem-style}
%   {\topsep} % Space above
%   {\topsep} % Space below
%   {\itshape} % Body font
%   {} % Indent amount
%   {\bfseries} % Theorem head font
%   {} % Punctuation after theorem head
%   {\newline} % Space after theorem head
%   {} % Theorem head spec (can be left empty, meaning `normal')
% \theoremstyle{theorem-style}
% \newtheorem{theorem}{Theorem}
% %\numberwithin{theorem}{subsection}
% \newtheorem{proposition}{Proposition}
% \newtheorem{corollary}{Corollary}
% \newtheorem{lemma}{Lemma}

% \newtheoremstyle{definition-style}
%   {\topsep} % Space above
%   {\topsep} % Space below
%   {} % Body font
%   {} % Indent amount
%   {\bfseries} % Theorem head font
%   {} % Punctuation after theorem head
%   {\newline} % Space after theorem head
%   {} % Theorem head spec (can be left empty, meaning `normal')
% \theoremstyle{definition-style}
% \newtheorem{definition}{Definition}
% \newtheorem{remark}{Remark}
% \newtheorem{example}{Example}


% Tables
\usepackage{booktabs,multirow,multicol}
%\renewcommand{\arraystretch}{1.5} % Increase table row height
\usepackage{csvsimple}
\usepackage{siunitx}


% Notes and Annotations
% \usepackage[prependcaption,textsize=small,color=gray!40]{todonotes} %option: disable


% Math operators and commands
\makeatletter
\newcommand{\superimpose}[2]{%s
  {\ooalign{$#1\@firstoftwo#2$\cr\hfil$#1\@secondoftwo#2$\hfil\cr}}}
\makeatother
\newcommand{\ostimes}{\mathbin{\mathpalette\superimpose{{\otimes}{\ominus}}}}
\newcommand{\oatimes}{\mathbin{\mathpalette\superimpose{{\otimes}{\obar}}}}
\makeatother
% to avoid warnings, copy only two symbols from stmaryrd
\DeclareSymbolFont{stmry}{U}{stmry}{m}{n}
\DeclareMathSymbol\leftarrowtriangle\mathrel{stmry}{"5E}
\DeclareMathSymbol\rightarrowtriangle\mathrel{stmry}{"5F}
\DeclareMathSymbol\sslash\mathrel{stmry}{"0C}
\DeclareMathSymbol\obar\mathrel{stmry}{"3A}
\DeclareMathSymbol\otimes\mathrel{stmry}{"0F}
\DeclareMathSymbol\ominus\mathrel{stmry}{"17}
\DeclareMathSymbol\minuso\mathrel{stmry}{"0A}

\renewcommand{\restriction}{\mathord{\upharpoonright}}
\renewcommand{\vec}{\operatorname{vec}}
\newcommand{\svec}{\operatorname{svec}}
\newcommand{\smat}{\operatorname{smat}}
\newcommand{\mat}{\operatorname{mat}}


% Colors
\usepackage{xcolor, colortbl}
\definecolor{lred}{RGB}{200,0,0}
\definecolor{dred}{RGB}{130,0,0}
\definecolor{dblu}{RGB}{0,0,130}
\definecolor{dgre}{RGB}{0,130,0}
\definecolor{dgra}{RGB}{50,50,50}
\definecolor{mgra}{RGB}{221,222,214}
\definecolor{lgra}{RGB}{238,238,234}
\definecolor{MPG}{RGB}{000,125,122}
\definecolor{lMPG}{RGB}{000,190,189}
\definecolor{ora}{HTML}{FF9933} %EI orange
\definecolor{lblu}{HTML}{7DA7D9}%PS blue
% Color scheme of the Eberhard-Karls University
\definecolor{TUred}{RGB}{141,45,57}
\definecolor{TUdark}{RGB}{55,65,74}
\definecolor{TUgold}{RGB}{174,159,109}
\definecolor{TUgray}{RGB}{175,179,183}
\definecolor{ERC_ora}{RGB}{233,93,15}
\setlength{\parindent}{0pt}

\newcommand{\mpg}[1]{{\color{MPG} #1}}   % highlight command 1
\newcommand{\dre}[1]{{\color{TUred} #1}}   % highlight command 1
\newcommand{\blu}[1]{{\color{dblu} #1}}   % highlight command 1
\newcommand{\ora}[1]{{\color{ora} #1}}   % highlight command 1
\newcommand{\gra}[1]{{\color{mgra} #1}}   % highlight command 1
\newcommand{\gold}[1]{{\color{TUgold} #1}}   % highlight command 1
\setbeamercolor{alerted text}{fg = TUred} % highlight command 2
\setbeamercolor{normal text}{fg=black,bg=white}
\setbeamercolor{structure}{fg=TUred}
\setbeamercolor{item projected}{use=item,fg=black,bg = TUred}
\setbeamercolor*{palette primary}{fg=white,bg=TUred}
\setbeamercolor*{palette secondary}{parent=palette primary,use=palette
primary,bg=dblu}
\setbeamercolor*{palette tertiary}{parent=palette
primary,use=palette primary,fg=white,bg=dgre}
\setbeamercolor*{palette
quaternary}{parent=palette primary,use=palette primary,bg=dgre}
\setbeamercolor*{block body}{bg=TUgray, fg =black}
\setbeamercolor*{block title}{parent=structure,bg=TUdark,fg=white}
\setbeamercolor{block body alerted}{bg=TUgray,fg=TUred}
\setbeamercolor{block title example}{fg=MPG,bg=white}
\setbeamercolor{block body example}{fg=black,bg=white}
\setbeamercolor{frametitle}{bg=TUdark,fg=white}
\setbeamercolor{frametitle right}{bg=white}
\setbeamercolor{framesubtitle}{fg=TUgray}
\setbeamercolor{title}{fg=TUred}
\setbeamercolor{subtitle}{fg=black} \setbeamercolor{author}{fg=TUdark}
\setbeamercolor{date}{fg=TUdark}
\setbeamercolor*{titlelike}{parent=structure}


% Beamer template settings and commands
\setbeamertemplate{navigation symbols}{}
\setbeamertemplate{bibliography item}[triangle]
\setbeamerfont{frametitle}{}
\let\emph\relax % redefine \emph to COLORS. % there's no \RedeclareTextFontCommand
\DeclareTextFontCommand{\emph}{\color{TUred}\bfseries} % changed from \em
\setbeamertemplate{itemize items}{\starredbullet}
\setbeamertemplate{footline}{\hfill\color{TUdark}{\insertframenumber}\hspace{2ex}\null\newline\vspace{2mm}}
\arrayrulecolor{TUdark}

\newcommand{\filltotal}{\hspace{0pt plus 1 filll}}
\newcommand{\titlemark}[1]{
  \begin{tikzpicture}[remember picture, overlay]
    \node[draw=none,text=TUgray,anchor=north east,yshift=-.8259cm] at (current
    page.north east) {\footnotesize{#1}};
  \end{tikzpicture}
}
\newcommand{\graybox}[1]{
  \tikzexternaldisable%
  \begin{center}%
    \tikz{\node[fill=TUdark,text width=\textwidth]{%
        \begin{minipage}{1.0\linewidth}%
        \color{white}
          \begin{center}%
            #1
          \end{center}%
        \end{minipage}};}%
  \end{center}%
  \tikzexternalenable
}
\newcommand{\divider}{\noindent\makebox[\linewidth]{\rule{\paperwidth}{.4pt}}  }
\newcommand{\paperwhite}{\includegraphics[height=.6\baselineskip]{../text-document_white.png}\hspace{.5em}}
\newcommand{\paperblack}{\includegraphics[height=.6\baselineskip]{../text-document.png}\hspace{.5em}}
\newcommand{\bookwhite}{\includegraphics[height=.6\baselineskip]{../book-white.png}\hspace{.5em}}
\newcommand{\bookblack}{\includegraphics[height=.6\baselineskip]{../book-black.png}\hspace{.5em}}

\setbeamercolor{ribboncolor}{fg=black,bg=TUgold}
\newcommand{\ribbon}[1]{
    \begin{beamercolorbox}[wd=\paperwidth,colsep*=.3em,center]{ribboncolor}
    \setbeamertemplate{itemize items}{\color{black}\starredbullet}
    \setbeamercolor{structure}{fg=white}
    \begin{minipage}{1.0\textwidth}%
            #1
        \end{minipage}
    \end{beamercolorbox}
}
\setbeamercolor{whiteribboncolor}{fg=black,bg=white}
\newcommand{\whiteribbon}[1]{
    \begin{beamercolorbox}[wd=\paperwidth,colsep*=.5em,center]{whiteribboncolor}
    \setbeamertemplate{itemize items}{\color{black}\starredbullet}
    \setbeamercolor{structure}{fg=black}
    \begin{minipage}{\textwidth}%
            #1
        \end{minipage}
    \end{beamercolorbox}
}
\newcommand{\bfa}[1]{
  \begin{beamercolorbox}[wd=\paperwidth,colsep*=.5em,center]{white}
    \setbeamertemplate{itemize items}{\color{black}\starredbullet}
    \tikzexternaldisable
    \begin{tikzpicture}
    \node[signal,minimum width=\paperwidth,draw=TUgold,fill=TUgold,text=black,text width=\textwidth]{
    \begin{minipage}{\textwidth}%
            #1
        \end{minipage}};
    \end{tikzpicture}
    \tikzexternalenable
    \end{beamercolorbox}
}
\newcommand{\bfi}[1]{
  \begin{beamercolorbox}[wd=\paperwidth,colsep*=.5em,center]{white}
    \setbeamertemplate{itemize items}{\color{black}\starredbullet}
    \tikzexternaldisable
    \begin{tikzpicture}
    \node[signal,signal from=west,signal to=nowhere,minimum width=\linewidth,draw=TUgold,fill=TUgold,text=black,signal pointer angle=140]{
    \begin{minipage}{\textwidth}%
            #1
        \end{minipage}};
    \end{tikzpicture}\hspace{2.5mm}
    \tikzexternalenable
    \end{beamercolorbox}
}
\newcommand{\blackslide}{
{\setbeamercolor{background canvas}{bg=black}
  \begin{frame}[plain]
    \null
  \end{frame}
}}
\newcommand{\blackslidetext}[1]{
{\setbeamercolor{background canvas}{bg=TUdark}%
  \setbeamercolor{structure}{fg=white}%
  \setbeamercolor{normal text}{fg=white}%
  \setbeamercolor{body}{fg=white}%
  \setbeamercolor{itemize/enumerate body}{fg=white}%
  \setbeamertemplate{itemize items}{\color{white}\starredbullet}%
  \begin{frame}
    \color{white} #1
  \end{frame}
}}
% logo in the upper right corner
\usepackage{eso-pic}
\newcommand\AtPagemyUpperLeft[1]{\AtPageLowerLeft{%
\put(\LenToUnit{0.8\paperwidth},\LenToUnit{0.915\paperheight}){#1}}}
\AddToShipoutPictureFG{
 % \AtPagemyUpperLeft{{\includegraphics[width=2.5cm,keepaspectratio]{assets/UT_WBMW_Weiss_1C.pdf}}}
  \AtPagemyUpperLeft{{\includegraphics[width=2.5cm,keepaspectratio]{assets/hku-white.png}}}

}%


% Tikz
\usepackage{etoolbox}
\usepackage{tikz}
\usetikzlibrary{tikzmark,arrows,shapes,plotmarks,decorations.pathmorphing}
\usetikzlibrary{backgrounds,calc,positioning,fadings}

% Tikz externalize
\usetikzlibrary{external}
\tikzexternalize[mode=list and make]
% use  "make -j 8 -f talk.makefile" to compile with 8 parallel threads (this is what it takes to max out the machine, despite it having 4 cores)
\tikzset{external/force remake=false}
\tikzsetexternalprefix{figures/external/}

\tikzset{>=stealth'}
\tikzstyle{graphnode} =
   [circle,draw=black,minimum size=22pt,text centered,text
     width=22pt,inner sep=0pt]
\tikzstyle{var}   =[graphnode,fill=white]
\tikzstyle{obs}   =[graphnode,fill=black,text=white]
\tikzstyle{act}   =[rectangle,draw=black,text=white,minimum
size=22pt,text centered, text width=22pt,inner sep=0pt]
\tikzstyle{fac}   =[rectangle,draw=black,fill=black!25,minimum size=5pt]
\tikzstyle{facprior} =[rectangle,draw=black,fill=black,text=white,minimum size=5pt]
\tikzstyle{edge}  =[draw=white,double=black,thick,-]
\tikzstyle{prior} =[rectangle, draw=black, fill=black, minimum size=
5pt, inner sep=0pt]
\tikzstyle{dirprior} = [circle, draw=black, fill=black, minimum
size=5pt, inner sep=0pt]

\tikzfading[name=fade top,bottom color=transparent!0,top color=transparent!75]
% to avoid warnings, copy only two symbols from stmaryrd
\DeclareSymbolFont{stmry}{U}{stmry}{m}{n}
\DeclareMathSymbol\leftarrowtriangle\mathrel{stmry}{"5E}
\DeclareMathSymbol\rightarrowtriangle\mathrel{stmry}{"5F}
\DeclareMathSymbol\sslash\mathrel{stmry}{"0C}
\DeclareMathSymbol\obar\mathrel{stmry}{"3A}
\DeclareMathSymbol\otimes\mathrel{stmry}{"0F}
\DeclareMathSymbol\ominus\mathrel{stmry}{"17}
\DeclareMathSymbol\minuso\mathrel{stmry}{"0A}
\renewcommand{\gets}{\operatorname*{\leftarrowtriangle}}
\renewcommand{\to}{\operatorname*{\rightarrowtriangle}}

\usetikzlibrary{arrows,shapes,plotmarks,pgfplots.colormaps}
\usetikzlibrary{pgfplots.groupplots}
\pgfplotsset{compat=newest}
\pgfplotsset{
  every axis legend/.append style =
    {
      cells = { anchor = east },
      draw  = none
    },
}
\makeatletter
\pgfplotsset{
    range frame/.style={
        tick align = outside,
        axis line style={opacity=0},
        after end axis/.code={
            \draw ({rel axis cs:0,0}-|{axis cs:\pgfplots@data@xmin,0}) -- ({rel axis cs:0,0}-|{axis cs:\pgfplots@data@xmax,0});
            \draw ({rel axis cs:0,0}|-{axis cs:0,\pgfplots@data@ymin}) -- ({rel axis cs:0,0}|-{axis cs:0,\pgfplots@data@ymax});
        }
    }
}
\makeatother

\pgfkeys{/pgfplots/mystyle/.style={
  % semithick,
  % tick style={major tick length=4pt,semithick,gray},
  xtick align = inside,
  ytick align = inside
  }}

\pgfkeys{/pgfplots/mytuftestyle/.style={
  semithick,
  tick style={major tick length=4pt,semithick,black},
  separate axis lines,
  axis x line*=bottom,
  axis x line shift=5pt,
  xlabel shift=0pt,
  axis y line*=left,
  tick align = outside,
  axis y line shift=5pt,
  ylabel shift=0pt}}


% Symbols
\newcommand{\cmark}{}%
\DeclareRobustCommand{\cmark}{%
  \tikz\fill[scale=0.4, color=black!30!green]
  (0,.35) -- (.25,0) -- (1,.7) -- (.25,.15) -- cycle;%
}
\newcommand{\xmark}{}%
\DeclareRobustCommand{\xmark}{%
  \tikz [x=1.4ex,y=1.4ex,line width=.2ex, color=red] \draw (0,0) -- (1,1) (0,1) -- (1,0);
}
\newcommand{\umark}{{\color{orange}\(\thicksim\)}}


% References and Bibliography
\usepackage{url}
\usepackage{xr}
\usepackage{hyperref}
\usepackage{cleveref}
\usepackage{bookmark} % Fixes false PDF table of contents
\usepackage{natbib}
\setcitestyle{authoryear, comma, numbers, sort&compress}
%\renewcommand*{\bibfont}{\small} % fontsize references
